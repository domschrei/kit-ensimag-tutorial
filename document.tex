\documentclass[11pt,a4paper]{article}
\usepackage[utf8]{inputenc}
\usepackage{german}
\usepackage{amsmath}
\usepackage{amsfonts}
\usepackage{amssymb}
\usepackage{textcomp}
\usepackage[width=17.00cm, height=24.00cm]{geometry}
\usepackage{tabularx}
\usepackage{graphics}
\usepackage{graphicx}
\usepackage{wrapfig}
\usepackage{tikz}

\begin{document}
	
	\begin{center}
	\huge Tipps für ein Auslandssemester in Grenoble\\
	\Large Für Informatiker am KIT, die zum ENSIMAG gehen (möchten)\\
	\vspace{1.5cm}
	\end{center}
			
	\tableofcontents
	
	\newpage
	\section{Einleitung}
	
	Zu Beginn unseres Auslandsaufenthalts hatten wir vieles zu entdecken und herauszufinden -- nicht nur einmal kam der Moment, an dem wir dachten, \glqq Mensch, wenn man das nur früher gewusst hätte...\grqq\ Um das ganze den folgenden Generationen etwas zu erleichtern, helfen hoffentlich die hier angesammelten Tipps und Erklärungen für einen möglichst reibungslosen Start und ein erfolgreiches Auslandsstudium!\\
	
	\noindent Dieses Dokument ist speziell für Informatikstudenten am KIT gedacht, die zu einem Auslandssemester/-jahr oder einem Doppelmaster-Studium zum ENSIMAG in Grenoble aufbrechen. Diverse Inhalte können aber auch für Studenten in einer anderen, ähnlichen Situation relevant sein.
		
	\newpage	
	\section{Vorbereitungen}
	
	\subsection{Bewerbung}
	
	Der allgemeine Bewerbungsprozess wird von der Informatikfakultät ausgeschrieben und organisiert, daher zunächst dringend deren Rahmenbedingungen und Deadlines einhalten. 
	
	Wird man vom KIT als Auslandsstudent angenommen, bewirbt man sich im nächsten Schritt beim Grenoble INP. Diese Bewerbung sollte, sofern man alle Dokumente fristgerecht (ausschließlich digital) einreicht, eine reine Formalität sein. 
	
	Parallel dazu bereitet man am KIT seine ERASMUS-Förderung vor -- dabei geht es vor allem um die Dokumente wie das Learning Agreement und (später) das Grant Agreement.
	
	\subsection{ERASMUS}
	
	In regelmäßigen ERASMUS-Infoveranstaltungen wird oft und ausführlich und genau erklärt, welche Dokumente wann, wie und wo einzureichen sind; darüber hinaus werden dieselben Informationen auch per eMail an die potentiellen Auslandsstudenten verstreut. 
	
	Einer der Knackpunkte dabei ist das Learning Agreement, denn hier geht es darum, welche Kurse man an der Partneruniversität belegen möchte und was man dafür am Ende am KIT angerechnet bekommt. Möchte man viele oder alle Kurse angerechnet bekommen (also insbesondere als Doppelmaster), ist ein frühzeitiges und umsichtiges Planen der Kurse nötig, damit sich nichts mit den bisher am KIT besuchten Kursen überschneidet und es tatsächlich für jeden Kurs am INP ein Pendant am KIT gibt, auf das die Leistung in Frankreich angerechnet werden kann. Es empfiehlt sich, mit einem möglichst konkreten Plan im Kopf (und auf dem Papier) zu einer Sprechstunde des Informatik-Studiengangsservices zu gehen und Rücksprache über die möglichen Optionen zu halten.
	
	\subsection{DFH (falls relevant)}
	
	Doppelmaster-Studenten müssen (oder besser: dürfen) sich zusätzlich noch bei der Deutsch-Französischen Hochschule registrieren. Diese \glqq virtuelle Hochschule\grqq\ vergibt feste finanzielle Förderungen an ihre Studenten und verleiht am Ende der Studien den Doppelmaster-Titel. 
	
	Auch hier unmittelbar nach der Annahme der Bewerbung durch das KIT nachfragen, was zu tun ist -- im Jahr 2017 lag die Deadline für die Immatrikulation bei der DFH bereits am 30.04., rückwirkend für das Studienjahr 2016/17 (denn eigentlich sollte man über den Zeitraum des gesamten Studiums, auch am KIT, bei der DFH eingeschrieben sein). Förderung erhält man aber nur über den Zeitraum des Auflandsaufenthalts hinweg.
	
	\subsection{Dokumente!}
		
	Auch wenn es nervig ist -- im Idealfall von allen offiziellen Dokumenten Kopien machen und am besten auch jeweils das ein oder andere ausgedruckte Exemplar dabeihaben. Eine kleine Checkliste, ohne Anspruch auf Vollständigkeit:
	
	\begin{itemize}
		\item Personalausweis
		\item Krankenkassenkarte
		\item Farbige Passfotos (Man braucht mehr davon, als man denkt, u.A. für Bahn-Monatskarten und Weiteres! Am besten eine farbige A4-Seite mit verschieden groß skalierten Passbildern an einem gewöhnlichen Farbdrucker drucken (lassen) und die Bilder zuschneiden.)
		\item Studentenausweis und Immatrikulationsbescheinigung(en) der Heimatuni
		\item Zulassungsbescheinigung aus Grenoble
		\item ERASMUS-Dokumente, insbesondere Learning Agreement, Grant Agreement und Certificate of Attendance
		\item Internationale Geburtsurkunde (in digitaler / eingescannter Form für die Beantragung des Wohnungsgelds nötig, und möglicherweise auch im Notfall praktisch zu haben)
	\end{itemize}

	\subsection{Anreise}
	
	Wer mit einem eigenen Auto kommen möchte: Es existieren außerhalb der Innenstadt, etwa auch bei den Wohnheimen am Campus (z.B. OUEST), ausreichend Parkplätze. 
	
	Für Autos, die den Großraum Grenoble befahren, gilt eine Pflicht für die französische Crit'Air Umweltplakette. Diese sollte einige Wochen vor Beginn der Abreise beantragt werden\footnote{https://www.certificat-air.gouv.fr/} -- es muss ein Scan des Fahrzeugscheins hochgeladen werden und man erhält bereits nach einigen Tagen einen Brief mit einem provisorischen Zertifikat, das als Ersatz für die eigentliche Plakette gilt. Letztere erhält man dann nach etwa drei Wochen ebenfalls per Post.
		
	\newpage
	\section{Erste Schritte}
	
	Die wichtigsten Knackpunkte vor Ort. (Das bedeutet nicht, dass man sich nicht bereits vor der Anreise darum kümmern sollte!)
	
	\subsection{Wohnort}
	
	Zweifellos einer der wichtigsten Punkte, die vor der Anreise zu klären sind: Wo wohne ich? Es stehen verschiedene Optionen zur Wahl.
	
	\subsubsection{CROUS-Wohnheime}
	
	Die studentischen Wohnheime stehen auch für internationale Studenten zur Verfügung. Wenn man sich hier rechtzeitig für eine Bleibe bewirbt, wird ein Platz in einem Wohnheim zugesichert. (Es sollte als erfolgreicher Bewerber am Grenoble INP spätestens im Juni eine eMail ankommen, die Informationen zur Online-Beantragung eines Wohnheimplatzes enthält.)
	
	Es handelt sich um eine vergleichsweise einfache Art der Wohnungsfindung, sofern alles glatt läuft, und durch den direkt bei der Schlüsselabholung ausgestellten Mietvertrag bzw. Wohnbescheinigung kann man potentiell direkt ein französisches Bankkonto eröffnen.
	
	Man kann sich nicht ein bestimmtes Wohnheim fest aussuchen, kann jedoch im Kommentarfeld des Antrags Präferenzen angeben. Folgende Faktoren kann man dabei berücksichtigen:
	
	\begin{itemize}
		\item \textbf{Lage:} Die Wohnheime auf dem Campus sind praktisch, falls man beispielsweise ausschließlich am ENSIMAG studiert, da man dann nur einen kurzen Weg zu den Vorlesungen hat. Dazu gehören etwa die Wohnheime OUEST, Hector Berlioz und Les Taillées. \\
		Studenten des GI (Génie Industriel) oder der UGA hingegen haben viele Veranstaltungen nahe des Bahnhofs, daher lohnt sich eher ein Wohnheim in der Stadt (etwa \emph{Maison des étudiants}).
		\item \textbf{Ausstattung:} Bei der Anmeldung für ein Wohnheim kann man verschiedene Arten der Beherbergung bevorzugen, etwa ein 12m$^2$ Zimmer mit Bad und Dusche und Etagen-Gemeinschaftsküche oder ein ähnlich großes Zimmer mit eigener kleinen Küchenzeile. Bei Gemeinschaftsküchen kommt es leider oft vor, dass Herdplatten ausfallen und ggf. zu den Haupt-Kochzeiten nicht alle Studenten, die kochen wollen, das zugleich auch können.
		\item \textbf{Kosten:} Je nach Ausstattung und Art des Wohnens kann man sich auf Kosten grob zwischen 200€ und 350€ einstellen. Zum Beispiel kostet ein 12m² Zimmer der Résidence OUEST mit eigenem Klo und Bad sowie gemeinschaftlich genutzter Küche etwa 280€ (Stand 09/17).
	\end{itemize}

	Spezielle Erwähnung verdient vielleicht noch das Wohnheim Rabot, denn dieses befindet sich auf einem Hügel auf halbem Weg zur Bastille. Das bedeutet zu jedem Verlassen und Betreten des Wohnheims erstmal eine kleine Wanderung! Auch die gemeinschaftlich genutzten Toiletten und die Ausstattung sind gewöhnungsbedürftig. Auf der anderen Seite ist das Wohnheim sehr günstig und einige Bewohner halten es für eine nette, weil sehr gemeinschaftliche, Bleibe.
	
	\subsubsection{Privates Wohnen}
	
	Es gibt natürlich die Möglichkeit, auf eigene Faust eine Bleibe zu suchen. Es gibt private Wohnheime (nicht von CROUS), die u.U. komfortabler, jedoch auch teurer sind. Möchte man in eine WG einziehen, findet jedoch von Deutschland aus noch nichts Konkretes, dann kann es sich anbieten, zunächst ein Studentenwohnheim-Zimmer anzunehmen und später beim erfolgreichen Finden einer WG vor Ort diese Bleibe wieder zu kündigen. Was allgemeine private Mietpreise angeht, kann man sich grob an den Preisen in Karlsruhe orientieren (natürlich stark schwankend nach Lage und wieteren Faktoren).
	
	\subsection{Bankkonto}
	
	In vielen Situationen ist es nützlich bis nötig, ein französisches Bankkonto zu besitzen. Einige Wohnheime können das Nachreichen der Kontodaten (in Form des RIB) verlangen, und auch im Alltag ist die Carte Bancaire nochmal an vielen Stellen einsetzbar, an denen die deutsche Kreditkarte unter Umständen versagt.
	
	Eine konkrete Option als Student des INP ist die Société Générale, eine der größten Banken in Frankreich, weil diese ein spezielles vergünstigendes Programm für diese Studenten anbietet (Stand 09/17). Man kann direkt die für das Wohnheim benötigte Haftpflicht-Versicherungen mit abschließen und geht durch einige Willkommensgeschenke am Ende mit mehr Geld als davor aus dem Geschäft (sofern das Konto und die Versicherung tatsächlich nur ein Jahr lang besteht).
	
	Um ein Bankkonto zu eröffnen, am Besten möglichst schnell persönlich zur gewünschten Bank gehen und um einen Termin bitten. Benötigte Dokumente sind eine Immatrikulations- (oder Zulassungs-) Bescheinigung, die Bescheinigung des Wohnortes und natürlich ein gültiger Personalausweis.
	
	\subsection{Versicherungen}
	
	Ganz abgesehen vom vernünftigen Bedürfnis, im Falle eines Unfalls oder Unglücks finanziell abgesichert zu sein, sind bestimmte Versicherungen als Student in Frankreich gesetzlich verpflichtend:
	
	\begin{itemize}
		\item Krankenversicherung: Als Patient einer gesetzlichen Krankenkasse in der EU ist man hier aus dem Schneider, da diese Versicherung EU-weit gültig ist (sichtbar an der europäischen Krankenkassenkarte). Als Privatpatient lieber nochmal nachfragen.\\
		Auch beachten: Die heimische Krankenkasse zahlt nur das Nötigste, und im schlimmsten Fall können Extremfälle wie eine Rücküberführung nach Deutschland ins Sechsstellige gehen, was selbst zu zahlen ist! Daher nach Möglichkeit noch eine ergänzende (Auslands)Krankenversicherung abschließen, entweder im Vorfeld in Deutschland oder in Frankreich durch eine \glqq Mutuelle\grqq\ wie LMDE oder SMERRA.
		\item Zivile Haftpflichtversicherung: Eine solche Versicherung kann man automatisch im Rahmen der Registrierung am ENSIMAG abschließen. Sie ist günstig (unter 20€ für das gesamte Jahr).
		\item Hausrats- bzw. Mieteigentums-Haftpflichtversicherung: Diese Versicherung wird insbesondere von CROUS verlangt, wenn man einen Wohnheimsplatz annimmt. Es ist jedoch in der Regel akzeptabel, eine entsprechende Bescheinigung zeitnah nach dem Einzug nachzureichen, da man für den Abschluss einer solchen Versicherung u.U. zunächst einen französischen Wohnsitz braucht! Eine solche Versicherung wird von einigen Banken im Rahmen der Eröffnung eines Bankkontos angeboten oder kann bei Institutionen wie LMDE oder SMERRA abgeschlossen werden.
	\end{itemize}
	
	\subsection{Registrierung an der Schule}
	
	Im Falle des ENSIMAG als Partnerhochschule (dies dürfte für alle Informatiker des KIT gelten) begibt man sich möglichst bald nach der Ankunft in Grenoble zum International Relations Office, in welchem man auf die hilfsbereite Aurélie Ducarre trifft (Stand 09/17 !). Hier bekommt man sein ERASMUS Certificate of Attendance unterschrieben und kann seine Registrierung am ENSIMAG vollziehen, sofern die benötigten Dokumente da sind (dazu sollte im Vornherein, also vor der Abfahrt nach Grenoble, eine eMail von Aurélie ankommen, in der die Registrierungsprozedur mit den benötigten Dokumenten genannt wird). Bringt man alles direkt von sich aus in ausgedruckter Form und fertig ausgefüllt mit, geht es am schnellsten.
	
	Den Studentenausweis erhält man erst, nachdem man die Studiengebühren (grob 40€, wenn man die eigene Krankenversicherung mitbringt) überwiesen oder anderweitig bezahlt hat und eine entsprechende Quittung vorweisen kann.
	
	Die Wahl der Kurse vor Ort ist mit der bloßen Abgabe des Learning Agreements noch nicht (zwangsläufig) getan! Im Beispiel des zweiten Masterjahrs etwa (relevant für Doppelmaster) muss man noch einmal über ein Online-Formular seine Kurswahl angeben, und erst dann wird ermittelt, ob überhaupt alle Vertiefungen angeboten werden können.
	
	\newpage
	\section{Weitere Schritte}
	
	Hier werden weitere Dinge zusammengefasst, die in Angriff genommen werden können, sobald das Nötigste vor Ort erledigt ist.
	
	\subsection{Wohnungsgeld -- CAF}
	
	Auf \texttt{www.caf.fr} kann man, sobald man ein entsprechendes Dokument von seinem Wohnheim erhalten hat und an seiner Schule immatrikuliert ist, das Wohngeld beantragen. Alle Dokumente sind digital einzureichen, insbesondere wird eine internationale Geburtsurkunde erwartet.
	
	Die Höhe der Förderung richtet sich nach dem persönlichen Wohnsituation und dem eigenen Vermögen.
	
	\subsection{Mobiltelefon}
	
	Es ist geschickt, eine französische Mobilfunknummer zu besitzen, denn auch mit EU-Roaming stößt irgendwann die Fair-Use-Policy an ihre Grenzen und es wird vergleichsweise teuer. Für eine unkomplizierte und günstige SIM-Karte ist der Anbieter \emph{free} zu empfehlen; in der Filiale kann man sich, Carte Bancaire vorausgesetzt, direkt an einem Automaten die SIM-Karte nach eigenen Wünschen herauslassen.
	
	\subsection{Öffentlicher Nahverkehr}
	
	Der TAG, der Grenobler Nahverkehr, bietet vergleichsweise unkomplizierte Tarife und Ticket-Optionen an. Es gibt nur die beiden Tarife \glqq unter 25\grqq\ und \glqq ab 25\grqq , keine Studententarife oder Sonstiges. Das Einzelticket kostet 1,60 (Stand 09/17) und gilt eine Stunde lang, egal in welche verschiedenen Richtungen und mit welchem TAG-Transportmittel man fährt. Ebenso gibt es Tickets für einen Tag und Zehnertickets.
	
	Monats- und Jahrestickets erfordern eine Karte im Scheckkartenformat mit Passbild (deshalb muss man diese an einem TAG-Schalter wie in der Innenstadt oder auch auf dem Campus bei \emph{Bibliothéques Universitaires} ausstellen lassen) für einmalig 5€, danach kostet jedes Monatsticket 15€ für Jugendliche unter 25. Für die Ausstellung der Karte ein Passbild dabeihaben und eine französische Adresse angeben können!
	
	\subsection{Izly}
	
	\newpage
	\section{Kleines Glossar}
	
	Für den Kampf durch den Dschungel der vielen unbekannten Abkürzungen hier eine kleine Zusammenstellung der wichtigsten, mitsamt Erklärung:\\
	
	\noindent \begin{tabularx}{\columnwidth}{|l|X|}
		\hline
		APL & \emph{Aide personnalisée au logement}; Wohnungsgeld, beanzutragen bei der CAF. \\ \hline
		CAF & \emph{Caisse d’allocations familiales}; bei dieser Institution kann man als Student (auch ausländisch) das Wohnungsgeld \emph{APL} beantragen. Scan einer internationalen Geburtsurkunde nötig! \\ \hline
		CB & \emph{Carte bancaire} (oder auch \emph{carte bleue}); die Bankkarten, die man bei Eröffnung eines französischen Bankkontos erhält, wie etwa bei der Société Générale. Nicht zu verwechseln mit deutschen Bankkarten oder Kreditkarten; diese müssen nicht überall dort funktionieren, wo Cartes bancaires unterstützt werden! \\ \hline
		CROUS & \emph{Centres régionaux des œuvres universitaires et scolaires}; vor allem relevant als Wohnungsgeber und Verwalter der Studentenwohnheime. \\ \hline
		CUEF & \emph{Centre universitaire d'études françaises} -- hier finden die Französisch-Sprachkurse (Intensivwoche sowie semesterbegleitend) statt. \\ \hline
		DFH & Deutsch-französische Hochschule; relevante Institution für Doppelmaster-Studenten, die finanziell fördert und Zeugnisse verleiht. \\ \hline
		ENSIMAG & \emph{École nationale supérieure d'informatique et de mathématiques appliquées de Grenoble}; Grande École, Teil des Grenoble INP. \\ \hline
		FLE & \emph{Français comme langue étrangère}; Bezeichnung für Französisch-Sprachkurse an französischen Universitäten für Internationale \\ \hline
		IMAG & Kurzform für ENSIMAG, insbesondere für verschiedene studentische und forschende Einrichtungen sowie die entsprechenden Gebäude verwendet \\ \hline
		INP & \emph{Institut (national) polytechnique (de Grenoble)}; Universitätssystem, bestehend aus mehreren Grandes Écoles, darunter das ENSIMAG. Ist Teil des Verbunds \emph{Univ-Grenoble-Alpes}, gehört aber nicht zur \emph{Université Grenoble-Alpes}. \\ \hline
		RIB & \emph{Relevé d'identité bancaire}; schriftliche Aufstellung der persönlichen (Überweisungs-)Daten eines französischen Bankkontos. Enthält IBAN und BIC, aber auch zusätzliche Identifikationsdaten der Bank usw.\\ \hline
		TAG & \emph{Transports de l'Agglomération Grenobloise}; der öffentliche Nahverkehr von Grenoble, also Trams und Busse, gehört dieser Institution an.\\ \hline
		UFA & \emph{Université Franco-Allemande}; französisches Synonym zu DFH. \\ \hline
		UGA & \emph{Université Grenoble-Alpes}; zum einen eine Universität (nicht Grande École) nicht zu werwechseln mit dem Grenoble INP; zum anderen auch ein Zusammenschluss von Höheren Bildungseinrichtungen, zu denen die erstgenannte Universität sowie auch das Grenoble INP angehören (in diesem Fall auch abgekürzt durch \emph{Univ-Grenoble-Alpes}). \\ \hline
	\end{tabularx}
	
\end{document} 
